% Included by MAIN.TEX
% Defines the settings for the CAMP report document

\renewcommand{\sectfont}{\normalfont \bfseries}        % Schriftart der Kopfzeile

% manipulate footer
\usepackage{scrpage2}
\pagestyle{scrheadings}
\ifoot[\footertext]{\footertext} % \footertext set in INFO.TEX
%\setkomafont{pagehead}{\normalfont\rmfamily}
\setkomafont{pagenumber}{\normalfont\rmfamily}

%% allow sophisticated control structures
\usepackage{ifthen}
\usepackage{longtable}

% use Palatino as default font
\usepackage{palatino}

% enable special PostScript fonts
\usepackage{pifont}

% make thumbnails
\usepackage{thumbpdf}

%to use the subfigures
%\usepackage{subfigure}

\usepackage{subcaption}

\usepackage{colortbl}

\usepackage{mathtools}


%% show program code\ldots
%\usepackage{verbatim}
%\usepackage{program}

%% enable TUM symbols on title page
\usepackage{styles/tumlogo}


\usepackage{multirow}

%% use colors
\usepackage{color}

%% make fancy math
\usepackage{amsmath}
\usepackage{amsfonts}
\usepackage{amssymb}
\usepackage{textcomp}
\usepackage{yhmath} % f�r die adots
%% mark text as preliminary
%\usepackage[draft,german,scrtime]{prelim2e}

%% create an index
\usepackage{makeidx}

% for the program environment
\usepackage{float}

%% load german babel package for german abstract
%\usepackage[german,american]{babel}
\usepackage[german,english]{babel}
\selectlanguage{english}

% use german characters as well
\usepackage[utf8]{inputenc}       % allow Latin1 characters

% use initals dropped caps - doesn't work with PDF
\usepackage{dropping}

\usepackage{styles/shortoverview}
%----------------------------------------------------
%      Graphics and Hyperlinks
%----------------------------------------------------

%% check for pdfTeX
\ifx\pdftexversion\undefined
 %% use PostScript graphics
 \usepackage[dvips]{graphicx}
 \DeclareGraphicsExtensions{.eps,.epsi}
 \graphicspath{{figures/}{figures/review}}
 %% allow rotations
 \usepackage{rotating}
 %% mark pages as draft copies
 %\usepackage[english,all,light]{draftcopy}
 %% use hypertex version of hyperref
 \usepackage[hypertex,hyperindex=false,colorlinks=false]{hyperref}
\else %% reduce output size \pdfcompresslevel=9
 %% declare pdfinfo
 %\pdfinfo {
 %  /Title (my title)
 %  /Creator (pdfLaTeX)
 %  /Author (my name)
 %  /Subject (my subject	)
 %  /Keywords (my keywords)
 %}
 %% use pdf or jpg graphics
 \usepackage[pdftex]{graphicx}
 \DeclareGraphicsExtensions{.jpg,.JPG,.png,.pdf,.eps}
 \graphicspath{{figures/}}

 %% Load float package, for enabling floating extensions
 \usepackage{float}

 %% allow rotations
 \usepackage{rotating}
 %% use pdftex version of hyperref
 \usepackage[pdftex,colorlinks=true,linkcolor=red,citecolor=red,%
 anchorcolor=red,urlcolor=red,bookmarks=true,%
 bookmarksopen=true,bookmarksopenlevel=0,plainpages=false%
 bookmarksnumbered=true,hyperindex=false,pdfstartview=%
 ]{hyperref}
%
%\usepackage[pdftex,colorlinks=false,linkcolor=red,citecolor=red,%
% anchorcolor=red,urlcolor=red,bookmarks=true,%
% bookmarksopen=true,bookmarksopenlevel=0,plainpages=false%
% bookmarksnumbered=true,hyperindex=false,pdfstartview=%
% ]{hyperref}
\fi




%% Fancy chapters
%\usepackage[Lenny]{fncychap}
%\usepackage[Glenn]{fncychap}
%\usepackage[Bjarne]{fncychap}

%\usepackage[avantgarde]{quotchap}

% set the bibliography style
%\bibliographystyle{styles/bauermaNum}
%\bibliographystyle{alpha}
\bibliographystyle{plain}


\DeclareMathOperator*{\argmin}{arg\,min}
\DeclareMathOperator*{\argmax}{arg\,max}



\usepackage{parskip}
%\setlength{\parskip}{3mm plus 4mm minus 4mm}
%\setlength{\parskip}{2mm plus0mm minus1mm}
%\setlength{\parindent}{0pt}
\clubpenalty = 10000
\widowpenalty = 10000



\usepackage{titlesec}
\titlespacing*{\section} {0pt}{2.5ex plus 1ex minus .2ex}{2.3ex plus .2ex}
\titlespacing*{\subsection} {0pt}{3.25ex plus 1ex minus .2ex}{1.5ex plus .2ex}
\titlespacing*{\subsubsection}{0pt}{3.25ex plus 1ex minus .2ex}{1.5ex plus .2ex}
\titlespacing*{\paragraph} {0pt}{3.25ex plus 1ex minus .2ex}{1em}
\titlespacing*{\subparagraph} {\parindent}{3.25ex plus 1ex minus .2ex}{1em}


\usepackage{listings}

\usepackage{pythonhighlight}

\usepackage{color}
\definecolor{mygreen}{rgb}{0,0.6,0}
\definecolor{mygray}{rgb}{0.5,0.5,0.5}
\definecolor{mymauve}{rgb}{0.58,0,0.82}

\usepackage{listings}
\usepackage{setspace}
\definecolor{Code}{rgb}{0,0,0}
\definecolor{Decorators}{rgb}{0.5,0.5,0.5}
\definecolor{Numbers}{rgb}{0.5,0,0}
\definecolor{MatchingBrackets}{rgb}{0.25,0.5,0.5}
\definecolor{Keywords}{rgb}{0,0,1}
\definecolor{self}{rgb}{0,0,0}
\definecolor{Strings}{rgb}{0,0.63,0}
\definecolor{argStrings}{rgb}{1,0.5,0}
\definecolor{Comments}{rgb}{0,0.63,1}
\definecolor{Backquotes}{rgb}{0,0,0}
\definecolor{Classname}{rgb}{0,0,0}
\definecolor{FunctionName}{rgb}{0,0,0}
\definecolor{Operators}{rgb}{0,0,0}
\definecolor{Background}{rgb}{0.98,0.98,0.98}
%TODO: when time remains: color numbers
%https://tex.stackexchange.com/questions/34896/coloring-digits-with-the-listings-package
\lstdefinelanguage{Python}{
numbers=left,
numberstyle=\footnotesize,
numbersep=1em,
xleftmargin=1em,
aboveskip=1em,
belowskip=1em,
framextopmargin=2em,
framexbottommargin=2em,
showspaces=false,
showtabs=false,
showstringspaces=false,
frame=L,
tabsize=4,
% Basic
basicstyle=\ttfamily\small\setstretch{1},
backgroundcolor=\color{Background},
% Comments
commentstyle=\color{Comments}\slshape,
morecomment=[l][\color{Strings}]{\#},
% Strings
stringstyle=\color{Strings},
morecomment=[s][\color{Strings}]{"""}{"""},
morecomment=[s][\color{Strings}]{'''}{'''},
morecomment=[s][\color{argStrings}]{'}{'},
% keywords
morekeywords={import,from,class,def,for,while,if,is,in,elif,else,not,and,or,print,break,continue,return,True,False,None,access,as,del,except,exec,finally,global,import,lambda,pass,print,raise,try,assert},
keywordstyle={\color{Keywords}\bfseries},
% additional keywords
morekeywords={[2]@invariant,pylab,numpy,np,scipy,tensorflow},
keywordstyle={[2]\color{Decorators}\slshape},
emph={self},
emphstyle={\color{self}\slshape},
%
}

\usepackage{xcolor}
\usepackage{caption}
\captionsetup{format=plain}

%use this for quoting
\usepackage{csquotes}

% Algorithms settings
\usepackage{algorithm}
\usepackage{algcompatible}
\usepackage[noend]{algpseudocode}
\newcommand\Let[2]{\State #1 $\gets$ #2}
\newcommand\Textline[1]{#1}
\makeatletter
\algnewcommand{\LineComment}[1]{\Statex \hskip\ALG@thistlm \(\triangleright\) #1}
\algnewcommand{\LineCommentFunc}[1]{\Statex \hspace{\leftmargin}\hspace{-1pt}\(\triangleright\) #1}
\makeatother
\newcommand\Blet[2]{\State \textbf{let} #1 \textbf{be} #2}
\errorcontextlines\maxdimen
% begin vertical rule patch for algorithmicx
% borrowing from http://tex.stackexchange.com/questions/41956/marking-conditional-versions-with-line-in-margin
% see http://tex.stackexchange.com/questions/110431/ploblems-with-vertical-lines-in-algorithmicx
\RequirePackage{zref-abspage}
\RequirePackage{zref-user}
\RequirePackage{tikz}
%\RequirePackage{atbegshi}
\usetikzlibrary{calc}
\RequirePackage{tikzpagenodes}
\RequirePackage{etoolbox}
\makeatletter
\newcommand*\ALG@lastblockb{b}
\newcommand*\ALG@lastblocke{e}
\apptocmd{\ALG@beginblock}{%
    %\typeout{beginning block, nesting level \theALG@nested, line \arabic{ALG@line}}%
    \ifx\ALG@lastblock\ALG@lastblockb
        \ifnum\theALG@nested>1\relax\expandafter\@firstoftwo\else\expandafter\@secondoftwo\fi{\ALG@tikzborder}{}%
    \fi
    \let\ALG@lastblock\ALG@lastblockb%
}{}{\errmessage{failed to patch}}

\pretocmd{\ALG@endblock}{%
    %\typeout{ending block, nesting level \theALG@nested, line \arabic{ALG@line}}%
    \ifx\ALG@lastblock\ALG@lastblocke
        \addtocounter{ALG@nested}{1}%
        \addtolength\ALG@tlm{\csname ALG@ind@\theALG@nested\endcsname}%
        \ifnum\theALG@nested>1\relax\expandafter\@firstoftwo\else\expandafter\@secondoftwo\fi{\endALG@tikzborder}{}%
        \addtolength\ALG@tlm{-\csname ALG@ind@\theALG@nested\endcsname}%
        \addtocounter{ALG@nested}{-1}%
    \fi
    \let\ALG@lastblock\ALG@lastblocke%
}{}{\errmessage{failed to patch}}
\tikzset{ALG@tikzborder/.style={line width=0.5pt,black}}
\newcommand*\currenttextarea{current page text area}
\newcommand*{\updatecurrenttextarea}{%
    \if@twocolumn
        \if@firstcolumn
            \renewcommand*{\currenttextarea}{current page column 1 area}%
        \else
            \renewcommand*{\currenttextarea}{current page column 2 area}%
        \fi
    \else
        \renewcommand*\currenttextarea{current page text area}%
    \fi
}
\newcounter{ALG@tikzborder}
\newcounter{ALG@totaltikzborder}
\newenvironment{ALG@tikzborder}[1][]{%
    % Allow user to overwrite the used style locally
    \ifx&#1&\else
        \tikzset{ALG@tikzborder/.style={#1}}%
    \fi
    \stepcounter{ALG@totaltikzborder}%
    \expandafter\edef\csname ALG@ind@border@\theALG@nested\endcsname{\theALG@totaltikzborder}%
    \setcounter{ALG@tikzborder}{\csname ALG@ind@border@\theALG@nested\endcsname}%
    %\typeout{begin ALG border nesting level=\theALG@nested, tikzborder=\theALG@tikzborder, tlm=\the\ALG@tlm}%
    \tikz[overlay,remember picture] \coordinate (ALG@tikzborder-\theALG@tikzborder);% node {\theALG@tikzborder};% Modified \tikzmark macro
    \zlabel{ALG@tikzborder-begin-\theALG@tikzborder}%
    % Test if end-label is at the same page and draw first half of border if not, from start place to the end of the page
    \ifnum\zref@extract{ALG@tikzborder-begin-\theALG@tikzborder}{abspage}=\zref@extract{ALG@tikzborder-end-\theALG@tikzborder}{abspage} \else
        \updatecurrenttextarea
        \ALG@drawvline{[shift={(0pt,.5\ht\strutbox)}]ALG@tikzborder-\theALG@tikzborder}{\currenttextarea.south east}{\ALG@thistlm}%
        % If it spreads over more than two pages:
        \newcounter{ALG@tikzborderpages\theALG@tikzborder}%
        \setcounter{ALG@tikzborderpages\theALG@tikzborder}{\numexpr-\zref@extract{ALG@tikzborder-begin-\theALG@tikzborder}{abspage}+\zref@extract{ALG@tikzborder-end-\theALG@tikzborder}{abspage}}%
        \ifnum\value{ALG@tikzborderpages\theALG@tikzborder}>1
            \edef\nextcmd{\noexpand\AtBeginShipoutNext{\noexpand\ALG@tikzborderpage{\theALG@tikzborder}{\the\ALG@thistlm}}}%some pages need a border on the whole page
            \nextcmd
        \fi
    \fi
}{%
    \setcounter{ALG@tikzborder}{\csname ALG@ind@border@\theALG@nested\endcsname}%
    %\typeout{end ALG border nesting level=\theALG@nested, tikzborder=\theALG@tikzborder, tlm=\the\ALG@tlm}%
    \tikz[overlay,remember picture] \coordinate (ALG@tikzborder-end-\theALG@tikzborder);% node {\theALG@tikzborder};% Modified \tikzmark macro
    \zlabel{ALG@tikzborder-end-\theALG@tikzborder}%
    % Test if begin-label is at the same page and draw whole border if so, from start place to end place
    \updatecurrenttextarea
    \ifnum\zref@extract{ALG@tikzborder-begin-\theALG@tikzborder}{abspage}=\zref@extract{ALG@tikzborder-end-\theALG@tikzborder}{abspage}\relax
        \ALG@drawvline{[shift={(0pt,.5\ht\strutbox)}]ALG@tikzborder-\theALG@tikzborder}{ALG@tikzborder-end-\theALG@tikzborder}{\ALG@thistlm}%
    % Otherwise draw second half of border, from the top of the page to the end place
    \else
        %\settextarea
        \ALG@drawvline{\currenttextarea.north west}{ALG@tikzborder-end-\theALG@tikzborder}{\ALG@thistlm}%
    \fi
}
\newcommand*{\ALG@drawvline}[3]{%#1=from, #2=to, #3=value of \ALG@tlm/\ALG@thisthm
    \begin{tikzpicture}[overlay,remember picture]
        \draw [ALG@tikzborder]
            let \p0 = (\currenttextarea.north west), \p1=(#1), \p2 = (#2)
             in
            (#3+\fboxsep+.5\pgflinewidth+\x0,\y1+\fboxsep+.5\pgflinewidth)%-\fboxsep-.5\pgflinewidth
             --
            (#3+\fboxsep+.5\pgflinewidth+\x0,\y2-\fboxsep-.5\pgflinewidth)
            %node[midway,anchor=east] {\ALG@tikzbordertext}
        ;
    \end{tikzpicture}%
}
\newcommand{\ALG@tikzborderpage}[2]{%the whole page gets a border, #1=value of \theALG@tikzborder, #2=value of \ALG@tlm/\ALG@thistlm
    \updatecurrenttextarea
    \setcounter{ALG@tikzborder}{#1}%
    \ALG@drawvline{\currenttextarea.north west}{\currenttextarea.south east}{#2}%
    \addtocounter{ALG@tikzborderpages\theALG@tikzborder}{-1}%
    \ifnum\value{ALG@tikzborderpages\theALG@tikzborder}>1
        \AtBeginShipoutNext{\ALG@tikzborderpage{#1}{#2}}%
    \fi
    \vspace{-0.5\baselineskip}% Compensate for the generated extra space at begin of the page. No idea why exactly this happens.
}
\def\ALG@tikzbordertext{\the\ALG@tlm}
% end vertical rule patch for algorithmicx

% continuation indent patch, slightly extended from http://tex.stackexchange.com/questions/78776/forced-indentation-in-algorithmicx to support multiple paragraphs in one block
\makeatletter
\newlength{\ALG@continueindent}
\setlength{\ALG@continueindent}{2em}
\newcommand*{\ALG@customparshape}{\parshape 2 \leftmargin \linewidth \dimexpr\ALG@tlm+\ALG@continueindent\relax \dimexpr\linewidth+\leftmargin-\ALG@tlm-\ALG@continueindent\relax}
\newcommand*{\ALG@customparshapex}{\parshape 1 \dimexpr\ALG@tlm+\ALG@continueindent\relax \dimexpr\linewidth+\leftmargin-\ALG@tlm-\ALG@continueindent\relax}
\apptocmd{\ALG@beginblock}{\ALG@customparshape\everypar{\ALG@customparshapex}}{}{\errmessage{failed to patch}}
\makeatother
% end continuation indent patch


\usepackage{amsthm}
\theoremstyle{plain}
\newtheorem{thm}[equation]{Theorem}
\newtheorem{lem}[equation]{Lemma}
\newtheorem{prop}[equation]{Proposition}
\newtheorem{cor}[equation]{Corollary}
\theoremstyle{definition}
\newtheorem{defn}[equation]{Definition}
\newtheorem{exmpl}[equation]{Example}


\usepackage{tikz}
\usetikzlibrary{shapes,arrows,chains}
\usetikzlibrary{decorations.text}
\usetikzlibrary{calc}
\usepackage{caption}
\tikzstyle{block} = [rectangle, draw, thick, text width=6em, text centered, rounded corners, minimum width = 3em, minimum height = 3em]
\tikzstyle{block_big} = [rectangle, draw, thick, text width=8em, text centered, rounded corners, minimum width = 8em, minimum height = 8em]
\tikzstyle{line} = [draw, -latex', thick, align=center]
\tikzstyle{round} = [circle, draw, thick, text width=2em, text centered, minimum width = 2em, minimum height = 2em]
\tikzstyle{round_big} = [circle, draw, thick, text centered, minimum width = 5em, minimum height = 5em]
\tikzstyle{square} = [rectangle, draw, thick, text width=3em, text centered, minimum width = 3em, minimum height = 3em]
